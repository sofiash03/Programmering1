Oppgave 2
I oppgave 4 i kapittel 5 brukte du klassen java.util.Random. Metoden nextInt() er laget slikt at 
den gir et heltall i intervallet fra og med 0 og opp til den tallverdien du sender inn som 
argument. 
Klassen tilbyr også en metode nextDouble() som gir deg et desimaltall uniformt 
fordelt i intervallet (0,0, 1,0) 15. 
Denne metoden tar ingen argumenter.
Lag en klasse MinRandom som tilbyr følgende metoder:
public int nesteHeltall(int nedre, int ovre) // intervallet (nedre, ovre)
public double nesteDesimaltall(double nedre, double ovre) 
// intervallet >nedre, ovre> 15. 
Denne notasjonen betyr at 0,0 er med, mens 1,0 ikke er med.

Klassen skal ha et objekt av klassen java.util.Random som objektvariabel. 
Metodene foran skal implementeres ved at man bruker dette objektet til å 
generere neste tilfeldige tall. 
Resultatet skal så transformeres til det ønskede intervallet.
Prøv klassen ved å lage mange tilfeldige tall av begge typer 
og forsikre deg om at de ligger i de oppgitte intervallene



Opprett MinRandom-klassen:
Lag en ny Java-klasse kalt "MinRandom" som skal inneholde metodene for å generere 
tilfeldige heltall og desimaltall i ønskede intervaller.

Legg til en objektvariabel av typen java.util.Random:
Inne i MinRandom-klassen, opprett en objektvariabel av typen java.util.Random for 
å generere tilfeldige tall.

Implementer metoden nesteHeltall:
Lag en metode med følgende signatur:

java
public int nesteHeltall(int nedre, int ovre)
Denne metoden skal bruke java.util.Random-objektet til å generere et
tilfeldig heltall i intervallet mellom "nedre" (inkludert) og "ovre" (ekskludert).

Implementer metoden nesteDesimaltall:
Lag en metode med følgende signatur:

java
public double nesteDesimaltall(double nedre, double ovre)
Denne metoden skal bruke java.util.Random-objektet til å generere et 
tilfeldig desimaltall i intervallet mellom "nedre" (inkludert) og "ovre" (inkludert).

Test klassen MinRandom:
Opprett en testklasse eller en testmetode for å verifisere at MinRandom-klassen fungerer
som forventet. Generer mange tilfeldige heltall og desimaltall ved hjelp av
 MinRandom-metodene og forsikre deg om at tallene faller innenfor de angitte intervallene.

Dette vil gi deg en MinRandom-klasse som kan generere tilfeldige tall i spesifikke
intervaller, basert på java.util.Random-klassen.